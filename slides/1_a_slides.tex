\documentclass{beamer}
%
% Choose how your presentation looks.
%
% For more themes, color themes and font themes, see:
% http://deic.uab.es/~iblanes/beamer_gallery/index_by_theme.html
%
\mode<presentation>
{
  \usetheme{default}      % or try Darmstadt, Madrid, Warsaw, ...
  \usecolortheme{crane} % or try albatross, beaver, crane, ...
  \usefonttheme{structurebold}  % or try serif, structurebold, ...
  \setbeamertemplate{navigation symbols}{}
  \setbeamertemplate{caption}[numbered]
} 

\usepackage[english]{babel}
\usepackage[utf8x]{inputenc}

\title[AI]{Deep Learning/Machine Learning/Artificial Intelligence}
\author{Pawel Wocjan}
\institute{University of Central Florida}
\date{Spring 2020}

\begin{document}

\begin{frame}
  \titlepage
\end{frame}

\begin{frame}{Sources for Slides}

\begin{itemize}
\item  I have used materials from 

\url{https://skymind.ai/wiki/ai-vs-machine-learning-vs-deep-learning} 

for the overview of artificial intelligence.
\end{itemize}
\end{frame}

% Uncomment these lines for an automatically generated outline.
%\begin{frame}{Outline}
%  \tableofcontents
%\end{frame}

\section{Artificial intelligence/machine learning/deep learning}

\begin{frame}{AI/ML/DL}
\begin{itemize}
\item You can think of deep learning, machine learning and artificial intelligence as a set of Russian dolls nested within each other.

\url{https://en.wikipedia.org/wiki/Matryoshka_doll}

\item Deep learning is a subset of machine learning, and machine learning is a subset of AI, which is an umbrella term for any computer program that does something smart. \item In other words, all machine learning is AI, but not all AI is machine learning, and so forth.
\end{itemize}
\end{frame}

\begin{frame}{Artificial intelligence}
\begin{itemize}
\item John McCarthy, one of the founders of artificial intelligence, defined it as ``the science and engineering of making intelligent machines.''

\url{https://en.wikipedia.org/wiki/John_McCarthy_(computer_scientist)}

\item Here are a few other definitions of artificial intelligence:

\begin{itemize}
\item A branch of computer science dealing with the simulation of intelligent behavior in computers.

\item The capability of a machine to imitate intelligent human behavior.

\item A computer system able to perform tasks that normally require human intelligence, such as visual perception, speech recognition, decision-making, and translation between languages.
\end{itemize}
\end{itemize}
\end{frame}

\begin{frame}{Symbolic AI/GOFAI}
\begin{itemize}
\item Symbolic artificial intelligence is the term for the collection of all methods in artificial intelligence research that are based on high-level "symbolic" (human-readable) representations of problems, logic and search.

\item Symbolic artificial intelligence is often called GOFAI (``Good Old-Fashioned Artificial Intelligence'').

\item The programming language Prolog is an example of symbolic artificial intelligence 

\url{https://en.wikipedia.org/wiki/Prolog}.

\url{https://swish.swi-prolog.org/example/queens.pl}
\end{itemize}
\end{frame}

\begin{frame}[fragile]{Symbolic AI/GOFAI}
Roughly speaking, symbolic AI operates like this:
\begin{verbatim}
                    
Input ------->  +---------+
                |         |-------> Output
Rules ------->  +---------+

\end{verbatim}
\end{frame}

\begin{frame}{Machine learning}
\begin{itemize}
\item In 1959, Arthur Samuel, coined the machine learning and defined it as a ``field of study that gives computers the ability to learn without being explicitly programmed.''

\url{https://en.wikipedia.org/wiki/Arthur_Samuel}

\item Machine-learning programs, in a sense, adjust themselves in response to the data they’re exposed to (like a child that is born knowing nothing adjusts its understanding of the world in response to experience).
\end{itemize}
\end{frame}

\begin{frame}{Machine learning}

\begin{itemize}
\item Machine learning is dynamic and does not require human intervention to make certain changes. 

\item That makes it less brittle, and less reliant on human experts.
\end{itemize}

\end{frame}


\begin{frame}{Machine learning}

\begin{itemize}
\item Tom Mitchell provided a widely quoated, more formal definition of the algorithms studies in the machine learning field:

\medskip
\begin{quote}
A computer program is said to learn from experience $E$ with respect to some class of tasks $T$ and performance measure $P$ if its performance at tasks in $T$, as measured by $P$, improves with experience $E$.
\end{quote}
\end{itemize}

\end{frame}



\begin{frame}[fragile]{Machine learning}
\begin{itemize}
\item One aspect that separates machine learning from symbolic AI is its ability to modify itself when exposed to more data.

\item Instead of coding up rules that transform the input to output, a machine learning system comes up with the rules itself.

\begin{verbatim}

Input ------->  +---------+
                |         |-------> Rules
Output ------>  +---------+

\end{verbatim}
\item The learned rules can then be used to predict outputs for new unseen inputs.

\item We will make this more precise, especially in the context of supervised learning.
\end{itemize}
\end{frame}

\begin{frame}{Combining two AI approaches}
\begin{itemize}
\item One may think that symbolic AI (GOFAI) is somewhat ``boring'', while machine learning (in particular, deep learning) is ``cool.''

\item This is not the case. For instance, the recent research paperstries to combine both approaches:

\medskip
\url{https://arxiv.org/pdf/1904.12584.pdf}

\medskip
Here is a short description of the main ideas in MIT  Technology Review:

{\tiny \url{https://www.technologyreview.com/s/613270/two-rival-ai-approaches-combine-to-let-machines-learn-about-the-world-like-a-child/}}
\end{itemize}
\end{frame}

\begin{frame}{Overview}
\begin{itemize}
\item This concludes the high-level overview of artificial intelligence.

\item Let's look at the three types of machine learning: supervised/unsupervised/reinforcement
\end{itemize}
\end{frame}

%
%
%

\end{document}

